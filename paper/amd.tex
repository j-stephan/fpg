\section{Messungen auf AMD-GPUs}
\label{amd}

\subsection{Verwendete Hard- und Software}

Die hier gezeigten Benchmark-Ergebnisse wurden auf einem Rechner mit der
folgenden Hardware gemessen:

\begin{itemize}
    \item CPU: AMD Ryzen Threadripper 1950X
          \begin{itemize}
              \item 16 Kerne
              \item 32 virtuelle Kerne
              \item maximaler Takt: \SI{3,4}{\giga\hertz}
          \end{itemize}
    \item GPU: AMD Radeon RX Vega 64
          \begin{itemize}
              \item 64 Multiprozessoren
              \item 64 Kerne pro Multiprozessor (insgesamt \num{4096} Kerne)
              \item maximaler Takt: \SI{1536}{\mega\hertz}
              \item \SI{8}{\gibi\byte} HBM2-Speicher
              \item Speicherbusbreite: \SI{2048}{\bit}
              \item Speicherbandbreite: \SI{483,3}{\gibi\byte\per\second}
              \item Speichertakt: \SI{945}{\mega\hertz}
          \end{itemize}
      \item RAM: \SI{64}{\gibi\byte}
\end{itemize}

Das verwendete Betriebssystem war Ubuntu 16.04 mit der Linux-Kernel-Version
4.15. Für die \gls{gpgpu}-Programmierung kamen die mit der \gls{rocm}-Version
2.0.89 mitgelieferten HIP- und HC-Implementierungen zum Einsatz.

\subsection{zcopy}

\subsection{Reduction}

\subsection{N-Body}

\subsubsection{Implementierung}

Die theoretische Funktion~\ref{methoden:nbody:gpu:bodybodyinteraction}, die die
Interaktion zwischen zwei Körpern berechnet, wurde direkt in beiden Sprachen
umgesetzt. Durch den Einsatz von \gls{fma}-Operationen werden die benötigten
\gls{flops} für die Berechnung des Skalarprodukts sowie der Beschleunigung
verringert. Die inverse Wurzel wird durch die \texttt{rsqrt}-Funktion berechnet.
Die Quelltexte~\ref{anhang:hip:bodybodyinteraction} (HIP) und
\ref{anhang:hc:bodybodyinteraction} (HC) im Anhang dieser Arbeit zeigen die
konkrete Implementierung.

Die theoretischen Funktionen~\ref{methoden:nbody:gpu:tilecalculation} und
\ref{methoden:nbody:gpu:calcforces} wurden zusammengefasst, da erstere
nur aus einer kurzen Schleife besteht. Überdies wurde der Compiler angewiesen,
die Schleife auszurollen (siehe auch den nächsten Abschnitt). Diese
Implementierungen finden sich in den angehängten
Quelltexten~\ref{anhang:hip:forcecalculation} (HIP) bzw.
\ref{anhang:hc:forcecalculation} (HC).

\subsubsection{Optimierung und Auswertung}
