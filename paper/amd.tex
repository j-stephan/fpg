\section{Messungen auf AMD-GPUs}
\label{amd}

\subsection{Verwendete Hard- und Software}

Die hier gezeigten Benchmark-Ergebnisse wurden auf einem Rechner mit der
folgenden Hardware gemessen:

\begin{itemize}
    \item CPU: AMD Ryzen Threadripper 1950X
          \begin{itemize}
              \item 16 Kerne
              \item 32 virtuelle Kerne
              \item maximaler Takt: \SI{3,4}{\giga\hertz}
          \end{itemize}
    \item GPU: AMD Radeon RX Vega 64
          \begin{itemize}
              \item 64 Multiprozessoren
              \item 64 Kerne pro Multiprozessor (insgesamt \num{4096} Kerne)
              \item maximaler Takt: \SI{1536}{\mega\hertz}
              \item \SI{8}{\gibi\byte} HBM2-Speicher
              \item Speicherbusbreite: \SI{2048}{\bit}
              \item Speicherbandbreite: \SI{483,3}{\gibi\byte\per\second}
              \item Speichertakt: \SI{945}{\mega\hertz}
          \end{itemize}
      \item RAM: \SI{64}{\gibi\byte}
\end{itemize}

Das verwendete Betriebssystem war Ubuntu 16.04 mit der Linux-Kernel-Version
4.15. Für die \gls{gpgpu}-Programmierung kamen die mit der \gls{rocm}-Version
2.0.89 mitgelieferten HIP- und HC-Implementierungen zum Einsatz.

\subsection{zcopy}

\subsection{Reduction}

\subsection{N-Body}

\subsubsection{Implementierung}

Die theoretische Funktion~\ref{methoden:nbody:gpu:bodybodyinteraction}, die die
Interaktion zwischen zwei Körpern berechnet, wurde direkt in beiden Sprachen
umgesetzt. Durch den Einsatz von \gls{fma}-Operationen werden die benötigten
\gls{flops} für die Berechnung des Skalarprodukts sowie der Beschleunigung
verringert. Die inverse Wurzel wird durch die \texttt{rsqrt}-Funktion berechnet.
Die Quelltexte~\ref{anhang:hip:bodybodyinteraction} (HIP) und
\ref{anhang:hc:bodybodyinteraction} (HC) im Anhang dieser Arbeit zeigen die
konkrete Implementierung.

Die theoretischen Funktionen~\ref{methoden:nbody:gpu:tilecalculation} und
\ref{methoden:nbody:gpu:calcforces} wurden zusammengefasst, da erstere
nur aus einer kurzen Schleife besteht. Überdies wurde der Compiler angewiesen,
die Schleife auszurollen (siehe auch den nächsten Abschnitt). Diese
Implementierungen finden sich in den angehängten
Quelltexten~\ref{anhang:hip:forcecalculation} (HIP) bzw.
\ref{anhang:hc:forcecalculation} (HC).

\subsubsection{Optimierung und Auswertung}

Eine einfache Optimierung ist das Ausrollen der Schleife, die nacheinander die
Interaktionen berechnet. Dadurch erhöht sich der Registerbedarf pro Thread, der
Overhead, der durch Verzweigungsinstruktionen anfällt, wird jedoch verringert.
Dieser Effekt ist deutlich in den Abbildungen~\ref{amd:nbody:unroll-hc} und
\ref{amd:nbody:unroll-hip} zu sehen:  durch die Bestimmung eines besseren
Ausrollfaktors lassen sich in diesem Benchmark bei einer festen Kachelgröße von
$p = 256$ knapp \num{1000} GFLOPS mehr Durchsatz gewinnen. Anhand dieser Messung
wurde für den weiteren Verlauf der Messungen ein Ausrollfaktor von 8 festgelegt.

\begin{figure}
    \centering
    \begin{tikzpicture}
        \begin{axis}[
            title = {Ausrollen -- HC},
            xlabel = {Ausrollfaktor},
            ylabel = {GFLOPS},
            xmode = log,
            log basis x = 2,
            xmin = 1, xmax = 512,
            xtick = {1,2,4,8,16,32,64,128,256,512},
            log ticks with fixed point,
            ymajorgrids = true,
            xmajorgrids = true,
            grid style = dashed,
            legend cell align = left,
            legend pos = outer north east,
            no markers,
            /pgf/number format/.cd, use comma
        ]
            \addplot table [x = count, y = gflops-hc, col sep = semicolon]
                           {data/nbody-amd-unroll-524288.csv};
            \addlegendentry{$n = 524.288$} 

            \addplot table [x = count, y = gflops-hc, col sep = semicolon]
                           {data/nbody-amd-unroll-65536.csv};
            \addlegendentry{$n = 65.536$} 

            \addplot table [x = count, y = gflops-hc, col sep = semicolon]
                           {data/nbody-amd-unroll-8192.csv};
            \addlegendentry{$n = 8.192$} 
        \end{axis}
    \end{tikzpicture}
    \caption{Performanzgewinn durch das Ausrollen der Schleife (HC)}
    \label{amd:nbody:unroll-hc}
\end{figure}

\begin{figure}
    \centering
    \begin{tikzpicture}
        \begin{axis}[
            title = {Ausrollen -- HIP},
            xlabel = {Ausrollfaktor},
            ylabel = {GFLOPS},
            xmode = log,
            log basis x = 2,
            xmin = 1, xmax = 512,
            xtick = {1,2,4,8,16,32,64,128,256,512},
            log ticks with fixed point,
            ymajorgrids = true,
            xmajorgrids = true,
            grid style = dashed,
            legend cell align = left,
            legend pos = outer north east,
            no markers,
            /pgf/number format/.cd, use comma
        ]
            \addplot table [x = count, y = gflops-hip, col sep = semicolon]
                           {data/nbody-amd-unroll-524288.csv};
            \addlegendentry{$n = 524.288$} 

            \addplot table [x = count, y = gflops-hip, col sep = semicolon]
                           {data/nbody-amd-unroll-65536.csv};
            \addlegendentry{$n = 65.536$} 

            \addplot table [x = count, y = gflops-hip, col sep = semicolon]
                           {data/nbody-amd-unroll-8192.csv};
            \addlegendentry{$n = 8.192$} 
        \end{axis}
    \end{tikzpicture}
    \caption{Performanzgewinn durch das Ausrollen der Schleife (HIP)}
    \label{amd:nbody:unroll-hip}
\end{figure}

Der nächste performanzrelevante Faktor ist die Größe der Kacheln selbst. Aus den
in den Abbildungen~\ref{amd:nbody:tilesize-hc} und \ref{amd:nbody:tilesize-hip}
dargestellten Messergebnissen wird ersichtlich, dass die Kachelgröße für den
Benchmark weniger wichtig ist; relevante Unterschiede sind nur bei großen
Kachelgrößen und wenigen Elementen sichtbar. Für den weiteren Messverlauf wird
daher eine Kachelgröße von $p = 256$ angenommen.

\begin{figure}
    \centering
    \begin{tikzpicture}
        \begin{axis}[
            title = {Kachelgrößen -- HC},
            xlabel = {Kachelgröße},
            ylabel = {GFLOPS},
            xmode = log,
            log basis x = 2,
            xtick = {64,128,256,512,1024},
            xticklabel = {\xinttheiexpr2^\tick\relax},
            log ticks with fixed point,
            ymajorgrids = true,
            xmajorgrids = true,
            grid style = dashed,
            legend cell align = left,
            legend pos = outer north east,
            no markers,
            /pgf/number format/.cd, use comma
        ]
            \addplot table [x = tilesize, y = gflops-hc, col sep = semicolon]
                           {data/nbody-amd-tilesize-524288.csv};
            \addlegendentry{$n = 524.288$} 

            \addplot table [x = tilesize, y = gflops-hc, col sep = semicolon]
                           {data/nbody-amd-tilesize-65536.csv};
            \addlegendentry{$n = 65.536$} 

            \addplot table [x = tilesize, y = gflops-hc, col sep = semicolon]
                           {data/nbody-amd-tilesize-8192.csv};
            \addlegendentry{$n = 8.192$} 
        \end{axis}
    \end{tikzpicture}
    \caption{Performanz bei verschiedenen Kachelgrößen (HC)}
    \label{amd:nbody:tilesize-hc}
\end{figure}

\begin{figure}
    \centering
    \begin{tikzpicture}
        \begin{axis}[
            title = {Kachelgrößen -- HIP},
            xlabel = {Kachelgröße},
            ylabel = {GFLOPS},
            xmode = log,
            log basis x = 2,
            xtick = {64,128,256,512,1024},
            xticklabel = {\xinttheiexpr2^\tick\relax},
            log ticks with fixed point,
            ymajorgrids = true,
            xmajorgrids = true,
            grid style = dashed,
            legend cell align = left,
            legend pos = outer north east,
            no markers,
            /pgf/number format/.cd, use comma
        ]
            \addplot table [x = tilesize, y = gflops-hip, col sep = semicolon]
                           {data/nbody-amd-tilesize-524288.csv};
            \addlegendentry{$n = 524.288$} 

            \addplot table [x = tilesize, y = gflops-hip, col sep = semicolon]
                           {data/nbody-amd-tilesize-65536.csv};
            \addlegendentry{$n = 65.536$} 

            \addplot table [x = tilesize, y = gflops-hip, col sep = semicolon]
                           {data/nbody-amd-tilesize-8192.csv};
            \addlegendentry{$n = 8.192$} 
        \end{axis}
    \end{tikzpicture}
    \caption{Performanz bei verschiedenen Kachelgrößen (HIP)}
    \label{amd:nbody:tilesize-hip}
\end{figure}

Mit der experimentell ermittelten Konfiguration lässt sich ein direkter
Vergleich zwischen \gls{hc} und \gls{hip} anstellen. Die
Abbildung~\ref{amd:nbody:comparison} zeigt, dass die Performanz bei beiden
\gls{api}s nahezu identisch ist. Der Blick in den generierten Maschinen-Code
zeigt, dass der \texttt{hcc}-Compiler in der Lage ist, für beide Varianten ein
identisches Ergebnis zu erzeugen (siehe Quelltexte~\ref{amd:nbody:isahc} und
\ref{amd:nbody:isahip}).

\begin{figure}
    \centering
    \begin{tikzpicture}
        \begin{axis}[
            title = {Leistungsvergleich -- HC und HIP},
            xlabel = {$n$},
            ylabel = {GFLOPS},
            xtick = data,
            xmode = log,
            log basis x = 2,
            xticklabel = {\xinttheiexpr2^\tick\relax},
            ymajorgrids = true,
            xmajorgrids = true,
            grid style = dashed,
            legend cell align = left,
            legend pos = outer north east,
            no markers,
            /pgf/number format/.cd, use comma,
            ybar,
            width = 0.75\textwidth,
            scale only axis,
            ymin = 0, ymax = 13000,
            extra y ticks = 12583,
            extra y tick labels = {},
            extra y tick style={grid=major, major grid style={solid,thick,draw=red}},
            scaled y ticks = false,
            ylabel near ticks,
            xlabel near ticks
        ]
            \addplot table [x = n, y = gflops-hc, col sep = semicolon]
                           {data/nbody-amd.csv};
            \addlegendentry{HC} 

            \addplot table [x = n, y = gflops-hip, col sep = semicolon]
                           {data/nbody-amd.csv};
            \addlegendentry{HIP} 

            \addlegendimage{vegamax}
            \addlegendentry{Max.}
        \end{axis}
    \end{tikzpicture}
    \caption{Leistungsvergleich zwischen HC und HIP}
    \label{amd:nbody:comparison}
\end{figure}

\begin{figure}
    \begin{minipage}{0.5\textwidth}
        \centering
        \begin{minted}[fontsize=\footnotesize]{text}
ds_read2_b64 v[14:17], v10 offset1:1
v_add_u32_e32 v9, 64, v9
s_waitcnt lgkmcnt(0)
v_sub_f32_e32 v16, v16, v5
v_sub_f32_e32 v15, v15, v4
v_fma_f32 v18, v16, v16, s20
v_sub_f32_e32 v14, v14, v3
v_fma_f32 v18, v15, v15, v18
v_fma_f32 v18, v14, v14, v18
v_mul_f32_e32 v19, v18, v18
v_mul_f32_e32 v18, v18, v19
v_cmp_gt_f32_e32 vcc, s21, v18
v_mov_b32_e32 v19, s22
v_cndmask_b32_e32 v20, 1.0, v19, vcc
v_mul_f32_e32 v18, v18, v20
v_rsq_f32_e32 v18, v18
v_mov_b32_e32 v20, s23
v_cndmask_b32_e32 v21, 1.0, v20, vcc
v_mul_f32_e32 v18, v21, v18
v_mul_f32_e32 v17, v17, v18
v_fma_f32 v18, v14, v17, v11
v_fma_f32 v15, v15, v17, v12
v_fma_f32 v16, v16, v17, v13                   
        \end{minted}
        \captionof{listing}{Maschinencode des HC-Kernels}
        \label{amd:nbody:isahc}
    \end{minipage}
    %
    \begin{minipage}{0.5\textwidth}
        \centering
        \begin{minted}[fontsize=\footnotesize]{text}
ds_read2_b64 v[14:17], v9 offset1:1
v_add_u32_e32 v10, 64, v10
s_waitcnt lgkmcnt(0)
v_sub_f32_e32 v16, v16, v5
v_sub_f32_e32 v15, v15, v4
v_fma_f32 v18, v16, v16, s16
v_sub_f32_e32 v14, v14, v3
v_fma_f32 v18, v15, v15, v18
v_fma_f32 v18, v14, v14, v18
v_mul_f32_e32 v19, v18, v18
v_mul_f32_e32 v18, v18, v19
v_cmp_gt_f32_e32 vcc, s17, v18
v_mov_b32_e32 v19, s18
v_cndmask_b32_e32 v20, 1.0, v19, vcc
v_mul_f32_e32 v18, v18, v20
v_rsq_f32_e32 v18, v18
v_mov_b32_e32 v20, s19
v_cndmask_b32_e32 v21, 1.0, v20, vcc
v_mul_f32_e32 v18, v21, v18
v_mul_f32_e32 v17, v17, v18
v_fma_f32 v18, v14, v17, v11
v_fma_f32 v15, v15, v17, v12
v_fma_f32 v16, v16, v17, v13
        \end{minted}
        \captionof{listing}{Maschinencode des HIP-Kernels}
        \label{amd:nbody:isahip}
    \end{minipage}
\end{figure}

Beide \gls{api}s erreichen jedoch nicht einmal die Hälfte der theoretisch
möglichen FLOPS.
