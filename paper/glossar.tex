\makeglossaries

\newacronym{api}{API}{\textit{application programming interface}}
\newacronym{apu}{APU}{\textit{accelerated processing unit}}

\newacronym{cpu}{CPU}{\textit{central processing unit}}

\newacronym{dsp}{DSP}{\textit{digital signal processor}}

\newacronym{flops}{FLOPS}{\textit{floating-point operations per second}}
\newacronym{fma}{FMA}{\textit{fused multiply-add}}
\newacronym{fpga}{FPGA}{\textit{field programmable gate array}}

\newacronym{gpgpu}{GPGPU}{\textit{general purpose computation on graphics processing unit}}
\newacronym{gpu}{GPU}{\textit{graphics processing unit}}

\newacronym{hc}{HC}{\textit{Heterogeneous Compute API}}
\newacronym{hcc}{HCC}{\textit{Heterogeneous Compute Compiler}}
\newacronym{hip}{HIP}{\textit{Heterogeneous-Computing Interface for Portability}}
\newacronym{hpc}{HPC}{\textit{high-performance computing}}
\newacronym{hsa}{HSA}{\textit{Heterogeneous System Architecture}}

\newacronym{opencl}{OpenCL}{\textit{Open Compute Language}}
\newacronym{openmp}{OpenMP}{\textit{Open Multi-Processing}}

\newacronym{rocm}{ROCm}{\textit{Radeon Open Compute Platform}}

\newacronym{simd}{SIMD}{\textit{single-instruction, multiple data}}
\newacronym{simt}{SIMT}{\textit{single-instruction, multiple thread}}

\newglossaryentry{kernel}{name = Kernel,
                          description = {Programm, das auf einem Beschleuniger,
                                         wie etwa einer GPU, ausgeführt wird.},
                          plural = Kernel}
\newglossaryentry{host}{name = Host,
                        description = {Gerät, das einen Kernel auf dem Device
                                       ausführen lässt. Üblicherweise das Gerät,
                                       auf dem das Betriebssystem läuft, etwa
                                       ein PC oder ein Knoten auf einem
                                       Superrechner.},
                        plural = Hosts}
\newglossaryentry{device}{name = Device,
                          description = {Ein von der CPU zu unterscheidendes
                                         Gerät für Berechnungen (Beschleuniger).
                                         Im Kontext dieser Arbeit stets eine
                                         GPU.},
                          plural = Devices}
\newglossaryentry{singlesource}{name = {single-source compilation},
                                description = {Die Quelltexte für Host und
                                               Device können sich in derselben
                                               Datei befinden und werden
                                               gleichzeitig vom Compiler
                                               verarbeitet.}}
\newglossaryentry{splitsource}{name = {split-source compilation},
                               description = {Die Quelltexte für Host und
                                              Device werden getrennt
                                              verarbeitet. Der Host-Quelltext
                                              wird von einem normalen Compiler
                                              übersetzt, der Device-Quelltext
                                              von einem weiteren Compiler und
                                              unter Umständen erst zur Laufzeit
                                              des Programms.}}
\newglossaryentry{flop}{name = {FLOP},
                        description = {Fließkommaoperationen (engl.
                                        \textit{floating point operations}).},
                        plural = FLOPs}
